\newpage
\phantomsection
\addcontentsline{toc}{chapter}{Summary}
\noindent \textbf{\huge Summary}

\vspace{1.5cm}
Current state-of-the-art approaches to convert non-RDF to RDF data in 
a streaming environment focus more on the efficiency of the 
mapping process with minimal support for multi-stream processing. 
The \emph{join} operator is one such commonly used operator in multi-stream processing. 
The existing approaches in mapping engines for supporting simple multi-stream processing operators
are very limited.
They require either a \emph{window} with fixed size, which performs badly with 
changing stream rate, or consume the data from 
one of the input fully before applying multi-stream processing operators.
Furthermore, related works for improving the dynamicity of windows, 
requires a memory overhead of keeping complex stream statistics to adapt 
to the varying stream characteristics.

Therefore, we implemented a dynamic window support in RMLStreamer, which 
adapts its size according to the changing stream characteristics with
negligible memory overhead, low latency, and high throughput. We evaluated the dynamic window
under different workload with varying stream velocity. The results 
show that it achieves latency in the milliseconds, with higher 
throughput than the fixed size windows in all workload situations. 

\paragraph{Keywords:}

RDF, RMLStreamer, RML, Adaptive windows, Dynamic windows,
Stream joins, Multi-stream processing.


\newpage
\noindent \textbf{\huge Samenvatting}

\vspace{1.5cm}
Huidige state-of-the-art benaderingen om niet-RDF naar RDF data om te zetten
in een streaming omgeving focussen zich meer op de efficiëntie van het 
mapping proces met minimale ondersteuning voor multi-stream verwerkingsoperatoren.
De join operator is zo'n veelgebruikte operator in een multi-stream omgeving.
De bestaande aanpakken in mapping engines voor de ondersteuning van 
eenvoudige multi-stream processing operatoren zijn zeer beperkt.
Ze vereisen ofwel een window met een vaste grootte, dat slecht presteert bij 
veranderende stream snelheid, of houden de data van een van de inputs 
volledig in geheugen alvorens multi-stream verwerkingsoperatoren toe te passen.
Daarom hebben wij in RMLStreamer een dynamisch window geïmplementeerd, 
dat de grootte zich aanpast aan de veranderende stream-karakteristieken met 
verwaarloosbare geheugenoverhead, lage latency en hoge doorvoer.
We hebben het dynamische window geëvalueerd onder verschillende werklast situaties 
met variërende stream snelheid.
De resultaten laten zien dat het dynamische window een latency in milliseconden bereikt, 
met een hogere doorvoer dan de windows met vaste grootte in alle werklast situaties.

\paragraph{Keywords:}

RDF, RMLStreamer, RML, Adaptive windows, Dynamic windows,
Stream joins, Multi-stream processing.

