\chapter{Window operators}
\label{chap:window_operators}

Due to the continuous, and infinite characteristics of streaming data, 
it is impossible to process the whole data in memory. Therefore, 
stream processing engines utilizes a buffer to hold the most recent stream of 
records in memory. This buffer is often called \emph{windows}. Windows make it 
possible to apply streaming version of the traditional data processing operators such
as joins~\cite{grubjoin}, sorts and aggregations. 

According to Bugra Gedik~\cite{generic_window_sem}, windows could be categorized into 
three different types; \emph{tumbling} windows, 
\emph{sliding} windows~\cite{stream_standford,spade_stream}, and \emph{partitioned} windows. 
\emph{Tumbling} and \emph{sliding} windows are most commonly provided as 
the default windows by the different stream processing frameworks. 
They offer the most flexibility in customizing them with \emph{windows parameters} to fit 
most use cases. Partitioned windows are special variations of the \emph{tumbling} and \emph{sliding}
windows, where a window consists of differnt sub-windows of same size. This allows 
\emph{multiplexed processing} where independent computations are done across the 
different sub-windows at the same time. This allows for a higher throughput because of 
the parallelization of the processing done in the window~\cite{generic_window_sem}.

The behaviour of these \emph{windows} are configured through the specification of different 
\emph{policies}; \emph{count}, \emph{attribute-delta}, \emph{time}, and
\emph{punctuation}-based policies~\cite{generic_window_sem}. For this thesis, only 
\emph{time}-based policy is of interest, since we are dealing with stream data with 
timeliness. \emph{Time}-based policy can appear as both \emph{eviction} and \emph{trigger} 
policy. 

To understand the different \emph{windows} and \emph{policies}, we will elaborate them 
in the following sections. We will first describe the general mechanism of the 
different window and then, explain the semantic behaviour of these different 
windows under the different policies.  





