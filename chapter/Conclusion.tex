\chapter{Conclusion and Future Works}%
\label{chap:Conclusion and Future Works}


In this paper, we have presented an approach for Dynamic window 
which adapts its window size according to the stream rate of the 
input data sources. Our approach aims to fix the shortcomings of 
the existing state-of-the-art dynamic windowing approaches for 
join operator. We introduced a simple heuristic to adapt our 
window sizes dynamically without huge memory or computation overhead. 

We implemented our Dynamic window on top of the existing RMLStreamer, 
to evaluate its performance under a realistic processing environment. 
We adapted the benchmark framework as stated in~\cite{evalution_of_spe} to 
accurately evaluate the performance of our implementation against the 
standard fixed size Tumbling window under different stream characteristics. 

The results gathered from our evaluation concludes that our implementation 
of Dynamic window fares better than Tumbling window in terms of 
latency, throughput, and completeness with only a slight 
increase in CPU usage. Even though, we could not confidently conclude that 
the memory usage is lower in Dynamic window, our preliminary results indicate 
that it performs at worst case as bad as Tumbling window.   

Therefore, there are still areas of improvement to be made in terms of the
evaluation setups
and the dynamic algorithm. On the evaluation side, we could further increase 
the precision of our memory measurement by only counting the number of records
residing in the windows at any moment instead of the whole JVM heap of the RMLStreamer job. 
Furthermore, the evaluation could be done in the same benchmark pipelines as in~\cite{evalution_of_spe} 
to further evaluate the Dynamic window performance in a general stream processing use case instead 
of in the context of RDF mapping engines.  
Improvements on our dynamic approach could be achieved by allowing user to 
define other statistical approach 
to better calculate the threshold for adapting the window sizes. For example, one could provide 
RMLStreamer with an external function to calculate the metrics in Dynamic window similar to the approach 
in FnO~\cite{fno_ben}.
As an extension, a dynamic window based on \emph{count} could be of interest in use 
cases where \emph{timeliness} is not required but the number of elements is of importance.
Furthermore, utilizing mapping file to configure our window might also allow further
optimization in the mapping file for a more efficient mapping job as proposed in FunMap~\cite{funmap}. 


