\chapter{Data Stream Processing}
\label{chap:data_stream_processing}

With the advent of \emph{Big Data} era, large amount of data are being generated 
by the companies from their infrastructure; from the IoT sensors to the events 
generated by the customers using their online services. The data that 
are generated are by nature \emph{unbounded} and infinite since as long 
as the infrastructure is working, new data will be continuously generated. 

For example, the healthcare industry has also been transitioning to 
adopt the approach of data-drive diagnostics methods~\cite{hospital_diagnosis}. 
This transition is caused by the need to keep up with the increase in the amount 
of physiological data generated by the monitoring sensors attached to the 
patients~\cite{hospital_data_monitoring}. These physiological data are generated 
periodically at regular intervals in a streaming manner. 

As a consequence of the large volume and continuous data generation, there is a need for 
the companies to be able to process these unbounded data streams. 
Stream processing engines are equipped to handle such unbounded data and exists in 
the industry today (e.g. Apache Flink~\cite{flink} and 
Apache Spark streaming~\cite{spark_streaming}). Main features of these stream 
processing engines are their ability to 
\renewcommand{\labelenumi}{(\roman{enumi})}
\begin{enumerate*}
    \item scale horizontally, 
    \item process records individually as they arrive, and
    \item process or analyse the data in real time.
\end{enumerate*}

In Chapter~\ref{chap:data_stream_processing}, we will dive into details of 
on the characteristics of data streams and also elaborate on the 
various state-of-the-art frameworks in use by the industries.

\section{Characteristics of Data Streams}
What is a data stream? As defined by Golab and Ozsu~\cite{golab_data_stream}:
“A \emph{data stream} is a \emph{real-time}, continuous, ordered (implicitly by arrival time 
or explicitly by timestamp) sequence of items. It is impossible to control the order
in which items arrive, nor is it feasible to \emph{locally store} a stream in its entirety.”
The difference between traditional Big Data processing and data stream processing, lies in 
the inability to store data \emph{locally} for data streams management. Furthermore, 
data streams processing requires real-time response. Therefore, there is a 
difference in the importance of the characteristics from the ones used to define Big Data. 

Big Data is usually characterized with the three “V's: \emph{volume}, \emph{velocity} 
and \emph{variety}~\cite{big_data_analytics}. 
In the context of data streams processing, it is expected that the \emph{volume} of 
the data is unbounded since the devices, sensors and applications are constantly 
generating data as long as there is an uptime. The \emph{variety} aspect of the 
data streams is also heterogeneous in nature where different sources used by the same 
processing engine may emit different types of data; structured and unstructured. 
Moreover, \emph{variety} in data can also happen over time; a concept drift where 
the properties of data may change over time unlike Big Data. 

Out of the three aforementioned characteristics, \emph{velocity} is the most important 
factor to consider when it comes to data stream processing. Data in a streaming 
environment is fast changing, and they arrive constantly into the stream processing
pipeline and needs to be processed in real-time. As a result of the need for real-time 
response, it is infeasible to have random memory access capability --- only single pass algorithm can
be applied. This requirement of low latency processing has a consequence 
that late decisions will result in missed opportunities. Related to \emph{velocity}, 
the data streams can also have a sudden \emph{'burst'} in the arrival rate which varies 
over time. 

In monitoring application of stream processing, the interest lies in the analysis of 
most recent data. As a result, the order of the data sequence needs to be considered 
when processing data streams. Due to the nature of data stream having to be transmitted over a 
network, the ordering of the data cannot be guaranteed. This is 
likely to happen if the network transmission protocols have no guarantee in the global 
ordering of data or if there is latency~\cite{requirements_dsp}. 

In the context of this work, where the focus is on the application of multi-stream operators, 
the \emph{velocity} of a data stream will have a huge impact on the quality and efficiency of 
the generated results. As a consequence of the limitations and challenges, brought forth by the 
aforementioned characteristics of data streams, a Data Stream Management System (DSMS) needs 
to be able to deal with them appropriately.  

\section{Data Stream Management Systems (DSMS)} 
Data Stream Management Systems are designed to overcome the aforementioned challenges 
in data stream processing. Different from the common big data processing, DSMS has to be 
able to respond in real-time and process the incoming data with low latency~\cite{data_stream_management}. 
For the former case, continuous queries are evaluated repeatedly by the DSMS to react 
to the incoming data. The latter could be tackled by reducing the inter-operator communications 
through optimizing operator scheduling and buffers to reduce network traffic~\cite{low_latency_data_stream}. 


