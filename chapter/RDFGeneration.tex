\chapter{RDF data generation}
Several state-of-the art implementations exist to generate RDF data from 
non-RDF data. These engines could be categorized into two groups; \emph{query-based} and 
\emph{rule-based} engines. These can be further categorized into two more subgroups 
based on the data that they consume; \emph{bounded} and \emph{unbounded} data processing 
engines. As mentioned in Chapter~\ref{chap:intro}, this work will focus on 
engines working with unbounded data in a streaming environment. 

\section{SPARQL Query Language}


\section{RDF Mapping Language}
RDF Mapping Language\cite{rml} is a superset of the W3C's R2RML\cite{r2rml} which maps relational databases to
RDF datasets. RML improves upon R2RML by expressing mapping rules from heterogeneous
data sources and transforming them to RDF datasets whereas R2RML could only consume
data from relational databases. RML mapping file is composed of one or more \emph{triples maps}, 
which in turn consist of \emph{subject, predicate} and \emph{object} term maps. As the names imply, 
the term maps are used to map elements of the data sources to their respective terms 
in an RDF triple. The definitions of these maps are similar to the 
specifications in R2RML\cite{rml_tech}. 

Logical sources could be defined by specifying the \emph{source, logical iterator} 
and zero or one \emph{reference formulation} property. The logical sources in the default 
RML mapping file are bounded data, where the data already exists and has a predetermined
size. RMLStreamer extends the vocabulary of RML to also handle unbounded data in 
a streaming context. 

RML supports defining relationships amongst the different 
logical sources through the use of \textit{rr:parentTriplesMap, rr:joinCondition, rr:child and rr:parent}
properties. The relationship definitions are


\begin{lstlisting}[caption=An example of a RML mapping file.]
    
    
\end{lstlisting}

\section{Mapping based Engine}
\subsection{RMLStreamer}
