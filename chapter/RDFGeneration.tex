\chapter{RDF data generation}
Several state-of-the art implementations exist to generate RDF data from 
non-RDF data. These engines could be categorized into two groups; \emph{query-based} and 
\emph{rule-based} engines. These can be further categorized into two more subgroups 
based on the data that they consume; \emph{bounded} and \emph{unbounded} data processing 
engines. As mentioned in Chapter~\ref{chap:intro}, this work will focus on 
engines working with unbounded data in a streaming environment. Before we dive into 
the engines, we need to elaborate more on the \emph{languages} these engines used 
to generate RDF data from non-RDF data. 


\section{SPARQL Query Language}
Query languages already exist in established relational database systems such as 
MySQL or PostgreSQL. It allows users to manipulate and retrieve data 
from relational databases using concise statements. Due to its widespread 
usage in the industry for querying databases, it is important that a similar 
query language is used for RDF datasets to ease the transition for the users. 

SPARQL\cite{sparql} achieves this goal by emulating as many SQL syntaxes as possible to allow 
a seamless transition to RDF datasets usage. Similar to relational databases, RDF 
datasets can be considered as a table consisting of three columns --- the subject column, 
the predicate column, and the object column. Unlike relational databases, 
RDF datasets, in a table representation, allows the object column to be of heterogeneous datatype. 
Recall from Chapter~\ref{sec:turtle_syntax}, one could explicitly specify the 
datatype of the object term which allows the heterogeneity in the object column.  
Also, different from SQL, SPARQL allows matching based on \emph{basic graph patterns} composed 
from a set of \emph{triple patterns}. Triple patterns are similar to the triple statements 
as clarified in Chapter~\ref{sec:turtle_syntax} but extended with declared variables (i.e. \emph{?variable\_name}). 
The result of a SPARQL query is returned as an RDF sub-graph of the queried RDF dataset. 

Now a question definitely gets raised in our mind, how does this relate to transforming an 
unbounded dataset to RDF format in a streaming environment? There exists state-of-the-art 
engines for generating RDF data from heterogeneous streaming data sources, which will be 
elaborated in Chapter~\ref{sec:query_based_engine}. 

\begin{lstlisting}[language=SPARQL,
    caption={Example of a SPARQL query of a medication.}, 
    label={lst:sparql_example}]
    SELECT ?medication
    WHERE {
        ?diagnosis example:name "Cancer" .
        ?medication example:canTreat ?diagnosis .
    }
\end{lstlisting}


\begin{table}[htbp]
    \centering
    \begin{tabular}{|c|}
    \hline
    medication        \\ \hline
    Radiation therapy \\ \hline
    \end{tabular}
    \caption{Result of executing the SPARQL query in Listing~\ref{lst:sparql_example}}
    \label{tab:sparql_result}
\end{table}





\section{RDF Mapping Language}
RDF Mapping Language\cite{rml} is a superset of the W3C's R2RML\cite{r2rml} which maps relational databases to
RDF datasets. RML improves upon R2RML by expressing mapping rules from heterogeneous
data sources and transforming them to RDF datasets whereas R2RML could only consume
data from relational databases. RML mapping file is composed of one or more \emph{triples maps}, 
which in turn consist of \emph{subject, predicate} and \emph{object} term maps. As the names imply, 
the term maps are used to map elements of the data sources to their respective terms 
in an RDF triple. The definitions of these maps are similar to the 
specifications in R2RML\cite{rml_tech}. 

Logical sources could be defined by specifying the \emph{source, logical iterator} 
and zero or one \emph{reference formulation} property. The logical sources in the default 
RML mapping file are bounded data, where the data already exists and has a predetermined
size. RMLStreamer extends the vocabulary of RML to also handle unbounded data in 
a streaming context. 

RML supports defining relationships amongst the different 
logical sources through the use of \textit{rr:parentTriplesMap, rr:joinCondition, rr:child and rr:parent}
properties. The relationship definitions are


\begin{lstlisting}[caption=An example of a RML mapping file.]
    
    
\end{lstlisting}

\section{Query based Engines}
\label{sec:query_based_engine}

\subsection{SPARQL-Generate}

\section{Mapping based Engines}
\subsection{RMLStreamer}
