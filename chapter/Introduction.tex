\chapter{Introduction}
\label{chap:intro}

A large volume of data is generated daily on the Web in a variety of domains. These
data are often structured according to an organization's specific needs or formats: Leading to
a difficulty in integrating the data across the different applications.
These generated data might have to be associated with archival data, also of heterogeneous formats,
to provide a coherent view required by analysis tasks. Heterogeneous Web data formats, such as CSV or HTML, are not explicitly
defined to enable linking entities in one document to other related entities in external documents.

Based on W3C standard, semantic data formats such as RDF triples~\cite{intro_rdf}, are a solution to
this particular problem by enriching the data with knowledge and association across
different domains, through the use of common ontologies. RDF triples also form the basic building blocks of knowledge graphs.
Knowledge graphs are extensively used in social networks like Facebook\cite{facebook_linked_data}, IoT devices\cite{graph_of_things} and especially with Google's search
engine\cite{google_kg}, it enables machines to understand the data and perform complex automated processing
using the knowledge graphs. 

Considering the aforementioned scenarios, there is a need to transform these non-RDF data to RDF compliant formats on the fly while
new data are being generated. Furthermore, we would also like to apply stream operators on the input tuples
before transforming, to enhance the enrichment of the data.

There exists state-of-the-art techniques to solve the task of consolidating heterogeneous data
and transforming them to an RDF compliant format. In this thesis, we will focus on one such format called TURTLE~\cite{turtle_syntax}.
These RDF transformation engines can be categorized into two major categories based on the type of input
which they consume; bounded and unbounded data input. Since we are focussing on the generation of RDF data
in a streaming environment, the class of RDF transformation engines on unbounded data will be of interest to our study.

Some engines support traditional stream operators like joins and aggregations. However, they do not consider
the characteristics of the streaming sources such as velocity and time-correlations between the different
input streams. This leads to a decline in the quality of the generated RDF triples. Moreover,
due to the nature of the infinite, continuous and real-time changing data of the streaming environment,
these operators have to be applied in the context of windows over a subset of the incoming data.
Clearly, with these restrictions and characteristics of the streaming sources, we need an adaptive approach
to applying these operators in windows for the data transformation engines. 

Therefore, we proposed a dynamic approach to windowing for multi stream operators.
The application of these operators in a dynamic window should maintain high memory efficiency and throughput,
even when the velocity of the streaming sources varies greatly over time. 
The dynamic window would make fast and simple statistical calculation to be 
aware of the velocity of the input streams to react and adapt its size accordingly. 
For implementing this dynamic window, we 
\renewcommand{\labelenumi}{(\roman{enumi})}
\begin{enumerate*}
    \item investigate the possible implementation sites which are near the \emph{input}
    or the \emph{output} of the engine, 
    \item provide a reference implementation utilizing the dynamic window, and 
    \item evaluating the implementation against widely used windowing schemes for 
    memory efficiency and throughput. 
\end{enumerate*}


This work aims to improve the performance multi-stream operators, 
by dynamically adapting the window size, according to the dynamic characteristics
of the incoming tuples for the multi-stream operators. Evaluation of 
the implementation is then intended to validate the following hypotheses: 

\begin{hyp}
    Dynamic windows maintain high memory efficiency than 
    fixed size windows even under a dramatic increase or 
    decrease in the velocity of incoming tuples. 
\end{hyp}

\begin{hyp}
    Higher throughput is achieved by multi-stream operators 
    with dynamic windows even with huge delays in the arrival
    rate of tuples between the different streams.
\end{hyp}

Naturally, to further build upon this work, a generic dynamic windowing 
framework could be specified for ease of evaluation and comparison between 
the different dynamic window schemes. Furthermore, we could also define a distributed 
statistical model to build up the characteristics of the streaming sources 
and to automatically apply the appropriate dynamic window. 


The rest of this work will be structured as follows. Chapter~\ref{chap:semantic_web} will 
go over the common terminologies and specifications related to Semantic Web. Readers familiar
with RDF framework and Semantic Web in general, could skip this chapter. 
Chapter~\ref{chap:data_stream_processing} will be about general data stream processing 
and how the state-of-the-art stream processing engines handle multi-stream operators. 
In Chapter~\ref{chap:rdf_data_generation}, current state-of-the-art in 
RDF data generation from non-RDF data sources will be discussed in detail. 
To understand more about windows in stream processing, Chapter~\ref{chap:window_operators}
will go into detail about the current fixed size windows and the other state-of-the-art 
dynamic window operators. 