\chapter{Introduction}
\label{chap:intro}

Large data volumes are generated daily on the Web in various domains.
Data integration is difficult across the different applications since these
data are often structured according to an organization's specific needs or formats. 
These generated data might have to be associated with heterogeneous archival data,  
to provide a coherent view for various applications such as 
data analysis. Heterogeneous data formats, such as CSV or HTML, are not
designed to link related entities across various datasets. 

Knowledge graphs are a solution to
this particular problem by enriching the data with knowledge and association across
different domains, through the use of common ontologies.
Knowledge graphs are also extensively used in social networks like 
Facebook\cite{facebook_linked_data}, IoT devices\cite{graph_of_things} and especially with Google's search
engine\cite{google_kg}, they enable machines to understand the data and perform complex automated processing
using knowledge graphs. 
Based on W3C standard, semantic data format such as Resource Description Framework (RDF)
triples~\cite{intro_rdf} is a possible serialization of knowledge graphs.  
RDF triples form the basic building blocks of knowledge graphs.

Considering the aforementioned scenarios, there is a need to transform these non-RDF data to RDF compliant formats on the fly while
new data are being generated. Furthermore, we would also like to apply stream operators on the input records
before transforming, to enhance the enrichment of the data.

There exists state-of-the-art approaches to consolidate heterogeneous data
and transform them to an RDF serialization. 
These RDF transformation engines can be categorized into two major categories based on the type of input
which they consume; \emph{bounded} and \emph{unbounded} data input. We are focusing on RDF transformation 
engines on unbounded data, as this thesis is centered around generation of RDF data in a streaming environment.  

Some engines support traditional stream operators like joins and aggregations. However, they do not consider
the characteristics of streaming data sources such as velocity and time-correlations between the different
input streams. This leads to a decline in the quality of the generated RDF triples. Because of the nature of the 
infinite, continuous and real-time changing data of the streaming environment, and 
the limited computing resources by the systems processing the streams, 
these operators have to be applied in the context of windows over a subset of the incoming data.
These operators are usually applied in fixed size windows,
Clearly, with these restrictions and characteristics of the streaming sources, there is a need 
for an adaptive approach to apply these operators in windows for the data transformation engines. 

Therefore, we propose a dynamic approach to windowing for multi stream operators.
The application of these operators in a dynamic window should maintain high memory efficiency and throughput,
even when the velocity of the streaming sources varies greatly over time. 
The dynamic window would make fast and simple statistical calculation to be 
aware of the velocity of the input streams to react and adapt its size accordingly. 
For implementing this dynamic window, we 
\renewcommand{\labelenumi}{(\roman{enumi})}
\begin{enumerate*}
    \item investigate the possible implementation sites which are near the \emph{input}
    or the \emph{output} of the engine, 
    \item provide a reference implementation utilizing the dynamic window, and 
    \item evaluate the implementation against widely used windowing schemes for 
    memory efficiency and throughput. 
\end{enumerate*}


This work improves the performance multi-stream operators, 
by dynamically adapting the window size, according to the dynamic characteristics
of the incoming records for the multi-stream operators. Evaluation of 
our approach is then intended to validate the following hypotheses: 

\begin{hyp}
    Dynamic windows consume lower memory than 
    fixed size windows even under a dramatic increase or 
    decrease in the velocity of incoming records. 
\end{hyp}

\begin{hyp}
    Dynamic window on operators enables lower latency 
    processing than fixed size windows. 
\end{hyp}

\begin{hyp}
    Dynamic windows enable the operators applied on
    unbounded data stream, to generate 
    results closer to the results of offline stream processing operations, 
    than fixed size windows. 
\end{hyp}
\begin{hyp}
    Higher throughput is achieved by multi-stream operators 
    with dynamic windows even with huge delays in the arrival
    rate of records between the different streams.
\end{hyp}

To further build upon this work, a generic dynamic windowing 
framework could be specified for ease of evaluation and comparison between 
the different dynamic window schemes. Furthermore, we could also define a distributed 
statistical model to build up the characteristics of the streaming sources 
and to automatically apply the appropriate dynamic window. 


The rest of this work will be structured as follows. Chapter~\ref{chap:semantic_web} will 
introduce common Semantic Web terminologies and specifications. Readers familiar
with RDF framework and Semantic Web in general, could skip this chapter. 
Chapter~\ref{chap:data_stream_processing} elaborates on general data stream processing 
and how the state-of-the-art stream processing engines handle multi-stream operators. 
Chapter~\ref{chap:rdf_data_generation} discusses state-of-the-art in 
RDF data generation from non-RDF data sources will be discussed in detail. 
To understand more about windows in stream processing, Chapter~\ref{chap:windows}
will go into detail about the current fixed size windows and the other state-of-the-art 
dynamic window operators. Chapter~\ref{chap:Methodology} describes the concepts and the 
implementation of our dynamic window followed by an evaluation of our implementation in 
Chapter~\ref{chap:Evaluation}. Chapter~\ref{chap:Discussion} discusses the results 
obtained from our evaluation, and the difficulty of evaluating the dynamic window in detail. 
Finally, Chapter~\ref{chap:Conclusion} concludes our work with possible extensions and 
improvements in the future. 
