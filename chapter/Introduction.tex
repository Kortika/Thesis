\chapter{Introduction}

A large volume of data is generated daily on the web in a variety of domains. These
data are often structured according to an organization's specific needs or formats: Leading to
a difficulty in integrating the data across the different applications.
These generated data might have to be associated with archival data, also of heterogeneous formats,
to provide a coherent view required by analysis tasks. Heterogeneous web data formats, such as CSV or HTML, are not explicitly
defined to enable linking entities in one document to other related entities in external documents.
Based on W3C standard, semantic data formats such as RDF triples~\cite{intro_rdf}, are a solution to
this particular problem by enriching the data with knowledge and associations across
different domains, through the usage of common ontologies. RDF triples also form the basic building blocks of knowledge graphs.
Knowledge graphs are extensively used in social networks like Facebook\cite{facebook_linked_data} and especially with Google's search
engine\cite{google_kg}, it enables machines to understand the data and perform complex automated processing
on the data. Therefore, there is a need to transform non-RDF data to RDF compliant format.

There have been recent state-of-the-art techniques to solve this task of consolidating heterogeneous data
and transforming them to a RDF compliant format. In this thesis, we will focus on one such format called RDF triples.
These RDF processing engines are


\section{Terminology and definitions}

Een sectie \cite{graph_of_things}.

\section{Nog een sectie}
\label{sec:nogeensectie}

Nog een.

\subsection{Een subsectie}
