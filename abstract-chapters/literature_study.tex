\section{RELATED WORKS} 
\label{sec:RELATED WORKS} 
There exist state-of-the-arts approaches to map non-RDF data 
to RDF data in a streaming environment. These approaches 
implement diferent strategies when it comes to applying 
multi-stream processing. Furthermore, we will also outline
the existing works on windowing to handle changing characteristics 
of the input data stream over time. 

\subsection{Mapping implementations}
\subsubsection{SPARQL-Generate} 
SPARQL-Generate~\cite{sparql_generate} 
is based on an extension of the SPARQL 1.1 query language, to leverage 
its expressiveness and extensibility. 
It can be implemented on top 
of any existing SPARQL query engine. 
M. Lefran\c{c}ois~\cite{sparql_generate} clarified that when joining
records from two different streams, SPARQL-Generate will fully consume the records 
from the "parent" stream and index it internally in memory first, before 
iteratively consuming the "child" stream for joining. 

\subsubsection{TripleWave} 
TripleWave~\cite{triple_wave}  is based on an extension of 
R2RML to consume heterogeneous data publish it as RDF data on the Web. 
Therefore, it only focuses on the mapping of a non-RDF data stream to an RDF data stream. 
Although it supports a simple \emph{join} operator, it does not have a 
dynamic window to handle changing characteristics of a data stream.

\subsubsection{RDFGen}
RDF-Gen~\cite{rdf_gen} is based on its own custom syntax 
which is a combination SPARQL and Turtle syntax. It processes the input streams 
individually and therefore employs windowing to perform multi stream operators. 
However, since the implementation is closed source, we could not confirm if 
the window is fixed size or dynamic. 

\subsubsection{RMLStreamer}
RMLStreamer~\cite{rml_streamer}
was developed to parallelize the ingestion and mapping process of RDF data generation pipeline. 
It is based on the work of RMLMapper~\cite{rml}, an RDF mapping engine consuming bounded data and mapping them to RDF data with the use of RML. Hence, RMLStreamer can also 
process heterogeneous data and generate RDF data. It does not support 
multi stream operators at this moment, however, the ease of scalability provided by RMLStreamer
is desirable as a stream mapping engine. 


\subsection{Windows}
Several studies have been conducted to 
improve join algorithms in windows~\cite{vctw_join, join_tracking, grubjoin, approximate_window_sem, approx_window}. 
The approaches in ~\cite{grubjoin, approximate_window_sem, approx_window} are based 
on dropping some of the records to be joined through \emph{load shedding} if the 
records fail to meet a threshold. These threshold calculations require some 
memory overhead for storing a statistical model per window or 
assume the stream to be in-order for them to work
~\cite{grubjoin, approximate_window_sem, approx_window}.

On the other hand, VC-TWJoin~\cite{vctw_join} considers only the velocity of the 
stream to adjust the window sizes. 
However, the approach is unstable when the data stream has periodic bursts of data.
It either increases or decreases the size of the window when the metric triggers 
the thresholds of the algorithm. In the worst-case scenario it 
can lead to cyclic increase and 
decrease of window sizes if the metric borders around the threshold with every 
update of the metric.




