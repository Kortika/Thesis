\section{INTRODUCTION}
\label{chap:intro}

Heterogene gegevensformaten, zoals CSV of HTML, zijn niet
ontworpen om de gegevens die zij representeren semantisch te verrijken en 
machines hebben moeite om deze gegevens automatisch te interpreteren. 
Om ervoor te zorgen dat deze heterogene gegevensformaten gemakkelijk 
door machines verwerkt en ge\"interpreteerd kunnen worden, 
worden gegevensformaten gebaseerd op W3C-normen, zoals Resource Description 
Framework (RDF) triples, ontwikkeld. 
Er bestaan state-of-the-art methodes om heterogene data te consolideren
en deze te transformeren naar een RDF serialisatie in een streaming omgeving. 

Deze benaderingen ondersteunen traditionele stream operatoren zoals joins en aggregaties. 
Echter, houden ze geen rekening met
de karakteristieken van streaming data bronnen zoals snelheid en 
tijd-correlaties tussen de verschillende
invoerstromen, hetzij als gevolg van de 
vaste grootte van windows of aangepaste oplossingen die zij toepassen. 

Daarom stellen wij een dynamisch window voor als oplossing voor de 
veranderende kenmerken van streaming data. Wij evalueerden onze 
methode met de join operator die toegepast werd binnen het venster. 
Het dynamische window verbetert de prestaties van de 
join operator met hogere doorvoer, lagere latency en 
vergelijkbaar geheugengebruik in vergelijking met een window met vaste grootte.

De broncode en de setup code van de evaluatie kunnen gevonden worden op
\url{https://github.com/RMLio/RMLStreamer/tree/feature/window_joins}, en 
\href{https://github.com/Kortika/Thesis-test-scripts.git}{https://github.com/Kortika/Thesis-test-scripts.git} respectievelijk.
