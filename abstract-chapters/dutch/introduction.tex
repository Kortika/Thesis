\section{INTRODUCTIE}
\label{chap:intro}

Heterogene gegevensformaten, zoals CSV of HTML, zijn niet
ontworpen om de gegevens die ze weergeven semantisch te verrijken en 
machines hebben moeite om deze gegevens automatisch te interpreteren.
Diverse gegevensformaten, zoals CSV of HTML, zijn niet ontworpen om de semantiek van hun gegevens te bevatten
waardoor machines deze gegevens niet automatisch kunnen interpreteren. 
Om ervoor te zorgen dat deze heterogene gegevensformaten interpreteerbaar en 
verwerkbaar zijn door machines, zijn gegevensformaten gebaseerd op W3C-normen, zoals Resource Description 
Framework (RDF) triples~\cite{intro_rdf}, worden ontwikkeld. 
Er bestaan state-of-the-art benaderingen om heterogene data te consolideren
en deze te transformeren naar een RDF serialisatie in een streaming omgeving. 

Deze benaderingen ondersteunen traditionele stream operatoren zoals joins en aggregaties. 
Echter, ze houden geen rekening met
de karakteristieken van streaming data bronnen zoals snelheid en 
tijd-correlaties tussen de verschillende
invoerstromen, hetzij als gevolg van de 
vaste grootte van windows of aangepaste oplossingen die zij toepassen. 

Daarom stellen wij een dynamisch venster voor als oplossing voor de variërende kenmerken van streaming data. Wij evalueerden onze 
benadering met de join operator toegepast binnen het venster. 
Het dynamische venster verbetert de prestaties van de 
join operator met hogere doorvoer, lagere latency en 
vergelijkbaar geheugengebruik vergeleken met een venster met vaste grootte.

De broncode en de code voor de evaluatieopstelling zijn te vinden op
\url{https://github.com/RMLio/RMLStreamer/tree/feature/window_joins}, en 

respectievelijk \href{https://github.com/Kortika/Thesis-test-scripts.git}{https://github.com/Kortika/Thesis-test-scripts.git}.
