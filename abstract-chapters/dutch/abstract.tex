%%%%%%%%%%%%%%%%%%%%%%%%%%%%%%%%%%%%%%%%%%%%%%%%%%%%%%%%%%%%%%%%%%%%%%%%%%%%%%%%
\begin{abstract}
De huidige state-of-the-art benaderingen om niet-RDF naar RDF data te converteren in 
een streaming omgeving richten zich meer op de efficiëntie van het 
mapping proces met minimale ondersteuning voor multi-stream verwerking. 
De bestaande benaderingen in mapping engines voor de ondersteuning van eenvoudige multi-stream 
verwerkingsoperatoren zijn zeer beperkt of passen een vaste window grootte toe.

Daarom hebben we in RMLStreamer een dynamisch venstermechanisme geïmplementeerd, dat 
de grootte aanpast aan de veranderende kenmerken van de stroom met
verwaarloosbare geheugenoverhead, lage latentie en hoge doorvoer. We evalueerden het dynamische venster
onder verschillende werkbelastingen met variërende streamsnelheid. De resultaten 
tonen aan dat het een latentie bereikt in het milliseconden bereik, met een hogere 
doorvoer dan vensters met een vaste grootte in alle werklast situaties. \\


\end{abstract}

\begin{keywords}
RDF, RMLStreamer, RML, Adaptieve windows, Dynamische windows,
Stream joins, Multi-stream processing.

\end{keywords}

%%%%%%%%%%%%%%%%%%%%%%%%%%%%%%%%%%%%%%%%%%%%%%%%%%%%%%%%%%%%%%%%%%%%%%%%%%%%%%%%
