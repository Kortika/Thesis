%%%%%%%%%%%%%%%%%%%%%%%%%%%%%%%%%%%%%%%%%%%%%%%%%%%%%%%%%%%%%%%%%%%%%%%%%%%%%%%%
\begin{abstract}

De huidige state-of-the-art benaderingen voor het mappen van niet-RDF naar RDF data in 
een streaming omgeving richten zich meer op de efficiëntie van het 
mapping proces met minimale ondersteuning voor multi-stream verwerking. 
De bestaande benaderingen voor de ondersteuning van eenvoudige multi-stream 
processing operatoren in mapping engines zijn zeer beperkt of passen een vaste venster grootte toe.

Daarom hebben we in RMLStreamer een dynamisch venstermechanisme geïmplementeerd, dat 
de grootte aanpast aan de veranderende kenmerken van de stroom met
verwaarloosbare geheugenoverhead, lage latentie en hoge doorvoer. We evalueerden 
het dynamische venster
onder verschillende werkbelastingen met variërende streamsnelheid. De resultaten 
tonen aan dat het een latentie bereikt in het milliseconden bereik, met een hogere 
doorvoer dan vensters met een vaste grootte in alle werklast situaties. \\ 
\end{abstract}

\begin{keywords}
RDF, RMLStreamer, RML, Adaptieve vensters, Dynamische vensters,
Stream joins, Multi-stream verwerking.

\end{keywords}

%%%%%%%%%%%%%%%%%%%%%%%%%%%%%%%%%%%%%%%%%%%%%%%%%%%%%%%%%%%%%%%%%%%%%%%%%%%%%%%%
