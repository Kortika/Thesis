\section{VERWANTE WERKEN} 
\label{sec:RELATED WORKS} 
Er bestaan state-of-the-arts benaderingen om niet-RDF data te mappen 
naar RDF data in een streaming omgeving. Deze benaderingen 
implementeren verschillende strategieën als het gaat om het toepassen van 
multi-stream verwerking. Verder schetsen we ook
de bestaande werken over windowing om veranderende karakteristieken 
van de input datastroom in de tijd. 

\subsection{Mapping implementaties}
\subsubsection{SPARQL-Generate} 
SPARQL-Generate~\cite{sparql_generate} 
is gebaseerd op een uitbreiding van de SPARQL 1.1 query taal, om gebruik te maken 
expressiviteit en uitbreidbaarheid te benutten. 
Het kan worden geïmplementeerd bovenop 
van elke bestaande SPARQL query-engine. 
M. Lefran\c{c}ois~\cite{sparql_generate} verduidelijkte dat bij het samenvoegen van
records van twee verschillende streams samenvoegt, SPARQL-Generate de records 
van de "ouder" stroom en deze eerst intern in het geheugen indexeren, voordat 
iteratief de "kind" stroom te consumeren voor het samenvoegen. 

\subsubsection{TripleWave} 
TripleWave~\cite{triple_wave} is gebaseerd op een uitbreiding van 
R2RML om heterogene gegevens te consumeren en als RDF-gegevens op het web te publiceren. 
Daarom richt het zich alleen op de mapping van een niet-RDF datastroom naar een RDF datastroom. 
Hoewel het een eenvoudige \emph{join} operator ondersteunt, heeft het geen 
dynamisch venster om veranderende karakteristieken van een datastroom te behandelen.

\subsubsection{RDFGen}
RDF-Gen~\cite{rdf_gen} is gebaseerd op zijn eigen aangepaste syntaxis 
die een combinatie is van SPARQL en Turtle syntax. Het verwerkt de invoerstromen 
individueel en maakt daarom gebruik van windowing om multistream operatoren uit te voeren. 
Echter, omdat de implementatie gesloten broncode is, konden we niet bevestigen of 
het venster een vaste grootte heeft of dynamisch is. 

\subsubsection{RMLStreamer}
RMLStreamer~\cite{rml_streamer}
werd ontwikkeld om het invoer- en mapping proces van RDF data generatie pijplijn te parallelliseren. 
Het is gebaseerd op het werk van RMLMapper~\cite{rml}, een RDF mapping engine die begrensde data consumeert en mapt naar RDF data met gebruik van RML. Daarom kan RMLStreamer ook 
heterogene gegevens verwerken en RDF-gegevens genereren. Het ondersteunt echter geen 
multi stream operators op dit moment, maar het gemak van schaalbaarheid dat RMLStreamer biedt
is wenselijk als een stream mapping engine. 


\subsection{Windows}
Er zijn verschillende studies uitgevoerd om 
join-algoritmen in windows ~\cite{vctw_join, join_tracking, grubjoin, approximate_window_sem, approx_window} te verbeteren. 
De benaderingen in ~\cite{grubjoin, approximate_window_sem, approx_window} zijn gebaseerd op 
op het laten vallen van sommige van de te verbinden records door middel van \emph{load shedding} indien de 
records niet aan een drempel voldoen. Deze drempelberekeningen vergen enige 
geheugenoverhead voor het opslaan van een statistisch model per venster of 
veronderstellen dat de stroom in orde is om te kunnen werken
~\cite{grubjoin, approximate_window_sem, approx_window}.

Anderzijds houdt VC-TWJoin~\cite{vctw_join} alleen rekening met de snelheid van de 
stroom om de venstergrootte aan te passen. 
Deze aanpak is echter onstabiel wanneer de gegevensstroom periodieke gegevensuitbarstingen vertoont.
De grootte van het venster wordt vergroot of verkleind wanneer de metriek de drempels 
de drempelwaarden van het algoritme. In het slechtste geval kan dit 
kan het leiden tot cyclische toename en 
verkleinen van de venstergrootte als de metriek bij elke 
update van de metriek.




