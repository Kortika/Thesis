\section{Conclusion and Future Works}%
\label{chap:Conclusion and Future Works}

In this paper, we have presented an approach for Dynamic window 
which adapts its window size according to the stream rate of the 
input data sources. We introduced a simple heuristic to adapt 
window sizes dynamically without huge memory or computation overhead. 

We implemented our Dynamic window on top of the existing RMLStreamer, 
to evaluate its performance under a realistic processing environment. 
We adapted the benchmark framework as stated in~\cite{evalution_of_spe} to 
accurately evaluate the performance of our implementation against the 
standard fixed size Tumbling window. 

The results show that our implementation 
of Dynamic window performs better than Tumbling window in terms of 
latency, throughput, and completeness with only a slight 
increase in CPU usage. Even though we could not confidently conclude that
memory usage is lower in Dynamic window, our preliminary results indicate 
that it performs the same as Tumbling window in the worst-case scenario.

Therefore, there are still areas of improvement to be made.
On the evaluation side, we could further increase 
the precision of our memory measurement by only counting the number of records
residing in the windows at any moment instead of the whole JVM heap of the RMLStreamer job. 
Furthermore, the evaluation could be done in the same benchmark pipelines as in~\cite{evalution_of_spe} 
to further evaluate the Dynamic window performance in a general stream processing case.
Improvements on our dynamic approach could be achieved by allowing users to 
define other statistical approaches
to better calculate the threshold for adapting the window sizes. 


