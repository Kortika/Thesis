\section{Conclusie en toekomstig werk}%
\label{chap:Conclusion and Future Works}

In deze paper hebben we een aanpak gepresenteerd voor Dynamisch venster 
die de venstergrootte aanpast aan de stroomsnelheid van de 
invoergegevensbronnen. We introduceerden een eenvoudige heuristiek om 
window-grootte dynamisch aan te passen zonder enorme geheugen- of rekenoverhead. 

We hebben ons dynamische venster geïmplementeerd bovenop de bestaande RMLStreamer, 
om de prestaties ervan in een realistische verwerkingsomgeving te evalueren. 
We hebben het benchmarkkader zoals beschreven in~\cite{evalution_of_spe} aangepast om 
nauwkeurig de prestaties van onze implementatie te evalueren ten opzichte van het 
standaard Tumbling-venster van vaste grootte. 

De resultaten tonen aan dat onze implementatie 
van Dynamisch venster beter presteert dan Tuimelend venster in termen van 
latentie, doorvoer, en volledigheid met slechts een kleine 
toename in CPU gebruik. Hoewel we niet met zekerheid kunnen concluderen dat
geheugengebruik lager is in Dynamisch venster, geven onze voorlopige resultaten aan 
dat het hetzelfde presteert als Tumbling window in het slechtste scenario.

Er zijn dus nog gebieden die voor verbetering vatbaar zijn.
Aan de evaluatiekant kunnen we de 
nauwkeurigheid van onze geheugenmeting kunnen verhogen door alleen het aantal records te tellen
in de vensters te tellen in plaats van de hele JVM-heap van de RMLStreamer-taak. 
Bovendien kan de evaluatie worden uitgevoerd in dezelfde benchmarkpijplijnen als in~\cite{evalution_of_spe} 
om de prestaties van het dynamische venster verder te evalueren in een algemeen stroomverwerkingsgeval.
Verbeteringen van onze dynamische aanpak kunnen worden bereikt door gebruikers in staat te stellen 
andere statistische benaderingen te definiëren
om de drempel voor het aanpassen van de venstergrootte beter te berekenen. 


